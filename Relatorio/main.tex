\documentclass{report}

\usepackage[portuguese]{babel}
\usepackage{graphicx}
\usepackage{listings, xcolor}
\usepackage[useregional]{datetime2}

\title{ACH2023 --- Algoritmos e Estruturas de Dados \\[1ex] \large Exercício Programa I}
\author{Igor Nascimento Silva}
\date{\DTMdisplaydate{2025}{12}{16}{-1}}

\begin{document}

\maketitle

\chapter{Estrutura}
O programa se divide em três arquivos principais: arvore.c, lista.c, 17076671.c.
Os arquivos \textbf{arvore.c} e \textbf{lista.c} contêm as funções padrão de suas
respectivas estruturas. A diferença mais notória nessas estruturas é a existência da variável
\textbf{ocorrencias} do tipo \textbf{Ocorrencia}, uma espécie de lista ligada
para cuidar de palavras duplicadas na leitura do arquivo.
Já o arquivo \textbf{17076671.c} é onde ``a mágica acontece'', é nele que ocorre a
leitura do arquivo, criação do índice e o loop principal do programa e, mesmo
que com alguma repetição, a criação de cada índice foi totalmente separada
para facilitar a leitura.

Em uma análise simples, o programa \textbf{valgrind} aponta não ter vazamentos
de memória possíveis nos três cenários que consegui pensar:
\begin{enumerate}
    \item Arquivo inválido
    \item Índice inválido
    \item Execução normal
\end{enumerate}
\begin{figure}[ht]
    \centering
    \includegraphics[width=0.4\textwidth, height=0.12\textheight]{./img/arquivo}
    \includegraphics[width=0.4\textwidth, height=0.12\textheight]{./img/indice}
    \includegraphics[width=0.4\textwidth, height=0.12\textheight]{./img/normal}
    \caption{Saída nos três casos.}
\end{figure}

\chapter{Resultados}
\section{Texto exemplo}
O texto exemplo é o trecho apresentado no enunciado do EP. Ele serve
como um texto pequeno, mas mostra exemplos esclarecedores. Aqui o uso
como indicado no enunciado usando os índices árvore e lista:
\begin{figure}[ht]
    \centering
    \includegraphics[width=0.48\textwidth, height=0.2\textheight]{./img/arvore1}
    \hspace{0.5cm}
    \includegraphics[width=0.48\textwidth, height=0.2\textheight]{./img/lista1}
    \caption{Exemplo mencionado.}
\end{figure}
Em média, as árvores fazem menos comparações, mas o arquivo é muito pequeno
notar alguma diferença considerável de tempo. A criação do índice é rápida
e a busca é mais rápida ainda.

\section{Crepúsculo dos ídolos: ou A filosofia a golpes de martelo}
Um texto que serve como uma entrada de tamanho médio. Agora é possível notar
uma pequena diferença de tempo na criação do índice, com a lista demorando um pouco
mais, mas a busca ainda é muito rápida. O número de comparações na lista já
se tornou por volta de 100x o número de comparações na árvores.

\begin{figure}[ht]
    \centering
    \includegraphics[width=0.40\textwidth, height=0.2\textheight]{./img/tempo2_arvore}
    \includegraphics[width=0.40\textwidth, height=0.2\textheight]{./img/tempo2_lista}
    \caption{Tempo de criação de índice e busca em cada índice.}
\end{figure}
\section{Memórias póstumas de Brás Cubas}
Outro texto que serve como uma entrada de tamanho médio, porém está aqui
para testar palavras com acentuação.\footnote{A ideia original era usar um
texto em japonês, mas o tratamento de caracteres Unicode e separação de
palavras se mostrou muito difícil.} Palavras acentuadas são diferentes
de palavras não acentuadas (e.g. ``está'' $\neq$ ``esta''), e tudo funciona normalmente.
O número de comparações na lista também é cerca de 100x o número de comparações na
árvore.
\begin{figure}[ht]
    \centering
    \includegraphics[width=0.40\textwidth, height=0.2\textheight]{./img/tempo3_arvore}
    \includegraphics[width=0.40\textwidth, height=0.2\textheight]{./img/tempo3_lista}
    \caption{Acentos funcionam como você esperaria.}
\end{figure}
\section{Anna Karenina}
Um texto que serve como uma entrada de tamanho grande. A criação de cada
indíce agora toma tempo considerável e a diferença entre a criação da
árvore e a da lista já não é mais marginal. A busca ainda é surpreendentemente
rápida para os dois índices. O número de comparações na lista é cerca de 1000x
o número de comparações na árvore.
\begin{figure}[ht]
    \centering
    \includegraphics[width=0.40\textwidth, height=0.2\textheight]{./img/tempo4_arvore}
    \includegraphics[width=0.40\textwidth, height=0.2\textheight]{./img/tempo4_lista}
    \caption{Diferença entre os índices se torna mais notória.}
\end{figure}
\section{Dicionário}
Um exemplo de entrada muito grande. O desempenho da lista é desproporcionalmente
pior do que o da árvore a ponto de, no primeiro momento, me questionar se
a demora se tratava de um \textit{bug}. Acredito que a explicação se encontra no fato
da lista ter de fazer muitas buscas na construção do índice e (quase) todas palavras
serem inseridas ao final dela, a árvore lidou muito melhor com o texto em ordem
alfabética. O tempo unitário da busca em cada índice continua magnificentemente rápido.
\begin{figure}[ht]
    \centering
    \includegraphics[width=0.40\textwidth, height=0.2\textheight]{./img/tempo5_arvore}
    \includegraphics[width=0.40\textwidth, height=0.2\textheight]{./img/tempo5_lista}
    \caption{Diferença desproporcional entre os índices.}
\end{figure}

\chapter{Conclusão}
O desempenho de cada índice depende mais do formato do arquivo de
entrada do que o tamanho dele. O maior gargalo do programa se encontra na
criação do índice, algo que só se torna útil caso o usuário queira fazer
sucessivas buscas no arquivo de entrada e faz o programa ficar extremamente
mais lento caso contrário.

Usei o dicionário como exemplo de resultado porque acreditva se tratar
do pior caso para os dois índices, mas essa ideia intuitiva se mostrou
falsa. Uma lista sequencial ao invés de uma lista ligada poderia se
beneficiar da busca no caso de uma entrada em ordem alfabética, porém é
um caso muito específico e a inserção ainda seria lenta.

\textbf{Fato curioso}: O programa usou, no máximo, $75.9$MB em ambos os índices (A árvore
usa um pouco mais de memória). Achei curioso porque, dado o tamanho da
entrada, pensava que seria muito mais. A título de comparação, o leitor de pdf
em que estou editando o relatório, que é conhecido por ser ``leve'' e ``minimalista''
(\textbf{Zathura}), usa cerca de $172.5$MB e o editor de texto (\textbf{Neovim})
usa $30.3$MB.

\end{document}
